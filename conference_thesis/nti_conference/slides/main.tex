
\documentclass{beamer}

\graphicspath{{./figure/}}

\usepackage[labelformat=empty,font=footnotesize]{caption}

\usepackage{cmap}

\usepackage{minted}
\setminted{fontsize=\tiny,baselinestretch=1}

\usepackage{verbatim}

\usepackage[T2A]{fontenc}
\usepackage[utf8x]{inputenc}
\usepackage[russian]{babel}

\usepackage{color}

\usepackage{comment}

\usetheme{Madrid}
\usecolortheme{beaver}

\title[Идентификация и автодифф.]{
	Применение автоматического дифференцирования при параметрической идентификации
	стохастических непрерывно-дискретных моделей
}
 
%\subtitle{A short story}
 
\institute[НГТУ, ФПМИ, ТПИ]{
	Новосибирский государственный технический университет (НГТУ) \\
	Факультет прикладной математики и информатики (ФПМИ) \\
	Кафедра теоретической и прикладной информатики (ТПИ)
}

\author[Горбунов К.]{
	Горбунов Константин, \tiny{магистрант, гр. ПММ-61} \\ 
	e-mail: gorbunov.2011@stud.nstu.ru \\
	\normalsize{Руководитель: Черникова Оксана Сергеевна, \tiny{доцент кафедры ТПИ}}
}
 
\date[НТИ-2017]{
	XI Всероссийская научная конференция молодых ученых \\ 
	<<Наука. Технологии. Инновации>> \\
	4 декабря -- 8 декабря~2017~года
}

% automatic slides / frames naming (titles, subtitles)
\begin{comment}
\addtobeamertemplate{frametitle}{
	\let\insertframetitle\insertsectionhead}{}
\addtobeamertemplate{frametitle}{
	\let\insertframesubtitle\insertsubsectionhead}{}

\makeatletter
	\CheckCommand*\beamer@checkframetitle{\@ifnextchar\bgroup\beamer@inlineframetitle{}}
	\renewcommand*\beamer@checkframetitle{
		\global\let\beamer@frametitle\relax\@ifnextchar\bgroup\beamer@inlineframetitle{}}
\makeatother
\end{comment}

\begin{document}


\frame{\titlepage}


\section*{Содержание}

\begin{frame}{\secname}{\subsecname}
	\tableofcontents[pausesections,pausesubsections]
\end{frame}


\section{Введение}

\subsection{Об идентификации}

\begin{frame}{\secname}{\subsecname}
	Сферы:
	\begin{itemize}
	  \item наука, техника
	  \item экономика, финансы
	\end{itemize}
	\medskip
	Смежные задачи:
	\begin{itemize}
	  \item планирование идентификационного эксперимента
	  \item оценивание состояния (фильтрация)
	  \item прогнозирование
	  %\item синтез (проектирование) % ?
	  \item управление
	\end{itemize}
	\medskip
	Предполагается считать тему работы актуальной.
\end{frame}


\begin{frame}{\secname}{\subsecname}
  Идентификация
  \begin{itemize}
	\item в широком смысле: \\
	  Структура модели неизвестна.
	\medskip
	\item в узком смысле: \\
	  Структура модели известна с точностью до некоторых неизвестных
	  параметров.
  \end{itemize}
  \medskip
\end{frame}


\section{Автоматическое дифференцирование (АД)}

\subsection{Предпосылки создания и применения метода АД}

\begin{frame}{\secname}{\subsecname}
  Рассмотрим задачу идентификации в широком смысле. \\

  Алгоритм идентификации: 
  \begin{enumerate}
	\item выбор структуры модели
	\item оценивание параметров
	\begin{itemize}
	  \item выбор критерия идентификации (ММП, МНК или др.)
	  \item выбор алгоритма оценивания отклика (и состояния) \\(фильтр Калмана
		или его модификация)
	  \item программирование вычисления критерия
	  \item программирование вычисления градиента критерия
	\end{itemize}
	\item проверка модели на адекватность: если адекватна, закончим процесс,
	  иначе вернуться к выбору структуры модели
  \end{enumerate}
  
\end{frame}

\begin{frame}{\secname}{\subsecname}
  Рассмотрим этап программирования вычисления градиента критерия.
  \begin{enumerate}
	\item вывод аналитического выражения градиента критерия (либо применить
	  конечно-разностный численный метод и закончить процесс)
	\item (непосредственно) программирование вычислений
  \end{enumerate}
\end{frame}

\begin{frame}{\secname}{\subsecname}
  Варианты:
  \begin{itemize}
	\item каждый раз выводить аналитически новый градиент, переписывать
	процедуру вычисления градиента критерия
  \item хранить процедуры вычисления градиентов <<на все случаи жизни>>
  \end{itemize}
  DRY --- Don't Repeat Yourself. \\
  Ответ: автоматическое дифференцирование
\end{frame}

\begin{frame}{\secname}{\subsecname}
  Justin Domke
\end{frame}


\subsection{Описание, определение}

\begin{frame}{\secname}{\subsecname}
  Автоматическое дифференцирование
  \begin{itemize}
	\item основано на представлении функции как суперпозиции элементарных
	  функций и арифметических операций, каждая из которых дифференцируема и
	  производная известна
	\item основано на правиле дифференцирования сложной функции (<<правиле
	  цепи>> --- <<chain-rule>>)
  \end{itemize}
\end{frame}


\subsection{Иллюстрация}

\begin{frame}{\secname}{\subsecname}
  Text
\end{frame}


\subsection{Доступные программные инструменты}

\begin{frame}{\secname}{\subsecname}
  Text
\end{frame}


\section{Постановка задачи}

\begin{frame}{\secname}
  Text
\end{frame}


\subsection{Описание модельной структуры}

\begin{frame}{\secname}{\subsecname}
  Text
\end{frame}


\subsection{Критерий идентификации}

\section{Результаты исследований}

\section{Дальнейшие планы исследований}

\begin{frame}
\begin{center}
  \LARGE{Спасибо за внимание.}
\end{center}
\end{frame}

\frame{\titlepage}

\end{document}
