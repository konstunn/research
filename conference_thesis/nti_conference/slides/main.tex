
\documentclass{beamer}

\graphicspath{{./figure/}}

\usepackage[labelformat=empty,font=footnotesize]{caption}

\usepackage{cmap}

\usepackage{minted}
\setminted{fontsize=\tiny,baselinestretch=1}

\usepackage{verbatim}

\usepackage[T2A]{fontenc}
\usepackage[utf8x]{inputenc}
\usepackage[russian]{babel}

\usepackage{color}

\usepackage{comment}

\usetheme{Warsaw}
\usecolortheme{seahorse}

\title[<<Цепочка обязанностей>>. Функциональный дизайн.]
{Шаблон <<Цепочка обязанностей>>. Шаблон~функционального дизайна.}
 
%\subtitle{A short story}
 
\author[Константин, Горбунов, ПММ-61]{Константин~Горбунов, ПММ-61}
 
\date[ТООП 2017]
{Технологии объектно-ориентированного программирования, Осень~2017}

\begin{document}

\frame{\titlepage}

\section{Цепочка обязанностей}
\subsection{Описание}
\begin{frame}
	Цепочка обязанностей (англ. Chain of resposibility) --- поведенческий
	объектный шаблон, предназначенный для организации в системе
	уровней ответственности.
	\medbreak \pause
	Поведенческие шаблоны
	\begin{itemize}
		\item описывают (не только) шаблоны классов и объектов
		\item (но и) описывают шаблоны взаимодействия между объектами
	\end{itemize}
	\medbreak \pause
	
	Шаблон <<цепочка обязанностей>> позволяет:
	\begin{itemize}
		\item посылать запросы / заявки / сообщения объектам неявно по цепочке
		объектов-кандидатам (получателям, обработчикам) на обработку запроса.
		\pause
		При этом:
		\begin{itemize}
			\item число объектов-кандидатов практически неограничено, \pause
			\item объекты-кандидаты можно добавлять в цепочку во время
			исполнения.
		\end{itemize}
	\end{itemize} \pause
	Цель шаблона: избежать <<зацепления>> отправителя запроса с получателем.
	Каждый получатель <<знает>> только одного (следующего по цепочке)
	получателя.
\end{frame}

\subsection{Когда использовать}
\begin{frame}
	Использовать когда: \pause \\
	\begin{itemize}
		\item несколько объектов могут выполнить запрос, и
		заранее неизвестно, какой объект будет выполнять запрос, это
		должно будет определиться автоматически \pause
		\item хотите послать запрос без явного указания исполнителя \pause
		\item множество объектов-исполнителей должно формироваться динамически
	\end{itemize}
\end{frame}

\subsection{Результаты применения}
\begin{frame}
	Результаты применения: \pause
	\begin{itemize}
		\item уменьшение <<зацепления>> \\ \pause
			Шаблон избавляет объект (отправителя запроса) от информации о том,
		какой другой конкретный объект обработает запрос. \pause
		\begin{itemize}
			\item упрощение взаимосвязей между объектами, вместо того чтобы
			каждому объекту хранить ссылки / указатели на каждого
			потенциального получателя, сохраняется ссылка только на следующий в
			цепочке объект \pause
		\end{itemize}
		\item гибкость назначения обязанностей объектам \pause
		\item \emph{\color{red}но нет гарантии, что запрос будет хоть кем-то
			обработан}\pause, потому как получатель заранее неизвестен
	\end{itemize}
\end{frame}

\subsection{Пример}
\begin{frame}
	Пример: \pause \\
	Для привлечения внимания... \pause

	Знакомьтесь. \pause
	\begin{figure}[!htb]
	\minipage{0.32\textwidth}
	\includegraphics[width=\linewidth]{riggs}
	\caption{Сержант Мартин Риггс, (<<Смертельное оружие>>, 1987)}
	\endminipage\hfill
	\pause
	\minipage{0.32\textwidth}
	\includegraphics[width=\linewidth]{hannah}
	\caption{Лейтенант Винсент Ханна, (<<Схватка>>, 1995)}
	\endminipage\hfill
	\pause
	\minipage{0.32\textwidth}%
	\includegraphics[width=\linewidth]{mcclain}
	\caption{Детектив Джон Макклейн, (<<Крепкий орешек>>)}
	\endminipage
	\end{figure}
	\pause
	Эти крутые полицейские будут расследовать преступления разной степени
	сложности.
\end{frame}

\begin{frame}[fragile]
\begin{minted}{cpp}
#include <iostream>

/**
 * Вспомогательный класс, описывающий некоторое преступление
 */
class CriminalAction {

	friend class Policeman;     // Полицейские имеют доступ к материалам следствия

	int complexity;             // Сложность дела

	const char* description;    // Краткое описание преступления

	public:
	CriminalAction(int complexity, const char* description):
		complexity(complexity), description(description) {}

};
\end{minted}
\end{frame}

\begin{frame}[fragile]
\begin{minted}{cpp}
/**
 * Абстрактный полицейский, который может заниматься расследованием
 преступлений
 */
class Policeman {
	protected:
		// дедукция (умение распутывать сложные дела) у данного полицейского
		int deduction;      

		// более умелый полицейский, который получит дело, если для текущего оно слишком сложное
		Policeman* next;    

		virtual void investigateConcrete(const char* description)
		{}    // собственно расследование

	public:
		Policeman(int deduction): deduction(deduction) {}

		virtual ~Policeman() {
			if (next) {
				delete next;
			}
		}

		/**
		 * Добавляет в цепочку ответственности более опытного полицейского,
		 * который сможет принять на себя расследование, если текущий не справится
		 */
		Policeman* setNext(Policeman* policeman) {
			next = policeman;
			return next;
		}
\end{minted}
\end{frame}

\begin{frame}[fragile]
\begin{minted}{cpp}
	/**
	 * Полицейский начинает расследование или, если дело слишком сложное,
	 * передает его более опытному коллеге
	 */
	void investigate(CriminalAction* criminalAction) {
		if (deduction < criminalAction->complexity) {
			if (next) {
				next->investigate(criminalAction);
			} else {
				std::cout << "Это дело не раскрыть никому." << std::endl;
			}
		} else {
			investigateConcrete(criminalAction->description);
		}
	}
};
\end{minted}
\pause
\begin{minted}{cpp}
class MartinRiggs: public Policeman {

	protected:

		void investigateConcrete(const char* description) {
			std::cout << "Расследование по делу \"" << description
				<< "\" ведет сержант Мартин Риггс" << std::endl;
		}

	public:

		MartinRiggs(int deduction): Policeman(deduction) {}
};
\end{minted}
\end{frame}

\begin{frame}[fragile]
\begin{minted}{cpp}
class JohnMcClane: public Policeman {

	protected:

		void investigateConcrete(const char* description) {
			std::cout << "Расследование по делу \"" << description
				<< "\" ведет детектив Джон Макклейн" << std::endl;
		}

	public:
		JohnMcClane(int deduction): Policeman(deduction) {}
};
\end{minted}
\pause
\begin{minted}{cpp}
class VincentHanna: public Policeman {

	protected:

		void investigateConcrete(const char* description) {
			std::cout << "Расследование по делу \"" << description
				<< "\" ведет лейтенант Винсент Ханна" << std::endl;
		}

	public:
		VincentHanna(int deduction): Policeman(deduction) {}
};
\end{minted}
\end{frame}

\begin{frame}[fragile]
\begin{minted}{cpp}
int main() {
	std::cout << "OUTPUT:" << std::endl;

	// полицейский с наименьшим навыком ведения расследований
	Policeman* policeman = new MartinRiggs(3);  

	policeman
		->setNext(new JohnMcClane(5))
		->setNext(new VincentHanna(8));     // добавляем ему двух опытных коллег
	policeman->investigate(new CriminalAction(2, "Торговля наркотиками из Вьетнама"));
	policeman->investigate(new CriminalAction(7, "Дерзкое ограбление банка в центре Лос-Анджелеса"));
	policeman->investigate(new CriminalAction(5, "Серия взрывов в центре Нью-Йорка"));
	return 0;
}

/**
 * OUTPUT:
 * Расследование по делу "Торговля наркотиками из Вьетнама" ведет сержант Мартин Риггс
 * Расследование по делу "Дерзкое ограбление банка в центре Лос-Анджелеса" ведет лейтенант Винсент Ханна
 * Расследование по делу "Серия взрывов в центре Нью-Йорка" ведет детектив Джон Макклейн
 */
\end{minted}
\end{frame}

\section{Шаблон функционального дизайна}
\subsection{Описание}
\begin{frame}[fragile]
Шаблон функционального дизайна \\
\medskip

Описание \pause
\begin{itemize}
	\item каждый модуль имеет одну обязанность \pause
	\item и при её исполнении дает минимум побочных эффектов на другие части
	программы \pause
	\item основной шаблон \pause
	\item идеализированный шаблон \pause
	\item не описан в книге GoF Design Patterns \pause
	\item уменьшение <<зацепления>> между частями \pause
	\item особенно актуален для проектирования систем 3D-моделирования
\end{itemize}

\end{frame}
\subsection{Преимущества}
\begin{frame}[fragile]
Преимущества
\begin{itemize}
	\item функционально спроектированную систему легче поддерживать \pause
	\item большее время жизни функционально спроектированной системы \pause
	\item повторное использование кода (DRY --- Don't Repeat Yourself)
\end{itemize}
\end{frame}

\begin{frame}[fragile]
\begin{center}
	Спасибо за внимание.
\end{center}
\end{frame}

\end{document}
