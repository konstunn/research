
\documentclass[a4paper,14pt]{extarticle}

\usepackage{cmap}

\usepackage[T2A]{fontenc}
\usepackage[utf8]{inputenc}
\usepackage[russian]{babel}

\usepackage[a4paper,margin=1cm,footskip=.5cm,left=2cm,right=1.5cm,top=1.5cm,bottom=1.5cm]{geometry}
\usepackage{textcase}
\usepackage{csquotes}
\usepackage{enumitem}
\usepackage[textsize=tiny]{todonotes}
\usepackage{titlesec}
\usepackage{amsmath}
\usepackage{indentfirst}
\usepackage{tabularx}

\setcounter{secnumdepth}{4}

\setlist[description]{leftmargin=\parindent,labelindent=\parindent}

\begin{document}

\begin{titlepage}

	\begin{center}
		\MakeTextUppercase{ Министерство образования и науки
			Российской~Федерации }
		\bigbreak
		ФЕДЕРАЛЬНОЕ ГОСУДАРСТВЕННОЕ БЮДЖЕТНОЕ ОБРАЗОВАТЕЛЬНОЕ УЧРЕЖДЕНИЕ
			ВЫСШЕГО ОБРАЗОВАНИЯ
		\bigbreak
		\MakeTextUppercase{\enquote{Новосибирский государственный технический
			университет}}
			\smallbreak
			\hrule
		\bigbreak\bigbreak

		Кафедра теоретической и прикладной информатики

		\vspace{100pt}

		\textbf{\LARGE{Дневник по}\\}
		\bigbreak
		учебной практике: \\
			практике по получению первичных профессиональных умений и навыков
		\bigbreak\bigbreak
		c 1 сентября по 31 октября

		\vspace{100pt}
	\end{center}

	\begin{center}
		\begin{tabular}{l}
			\medbreak
			Студент:         Горбунов Константин Константинович \\
			\medbreak
			Место практики:  Новосибирский государственный технический университет \\
			\medbreak
			Руководитель практики: доц. Черникова Оксана Сергеевна \\
			\medbreak
		\end{tabular}
	\end{center}

	\begin{center}
		\vspace{\fill}
		Новосибирск, 2016 г.
	\end{center}

\end{titlepage}

\newpage

\noindent \textbf{Тема учебной практики: практики по получению первичных профессиональных
умений и навыков:} робастные алгоритмы оценивания состояний линейных 
непрерывно-дискретных стохастических динамических систем.

\bigbreak

\noindent \textbf{Задание на практику:}
Провести обзор и сравнительный анализ робастных алгоритмов оценивания состояний
линейных непрерывно-дискретных стохастических динамических систем.

\begin{center}
\textbf{План-график выполнения работы}
\bigbreak
\begin{tabularx}{\linewidth}{|c|X|}
	\hline
	Дата выполнения & Краткое содержание выполненной работы \\ \hline
	01.09.2016 -- 19.09.2016 & Изучение литературы по теме исследований \\ \hline
	20.09.2016 -- 17.10.2016 & Исследование алгоритмов фильтрации применительно к
	непрерывно-дискретным моделям \\ \hline
	18.10.2016 -- 31.10.2016 & Оформление отчета \\ \hline
\end{tabularx}
\end{center}

\noindent Отметка руководителя о выполнении задания:

\end{document}
