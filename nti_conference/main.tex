
\documentclass[10pt,a5paper]{article}

%\usepackage{times}

\usepackage[top=20mm,bottom=20mm,left=19mm,right=19mm]{geometry}

\usepackage{lipsum}
\usepackage{blindtext}

\pagenumbering{gobble}

\setlength\parindent{0.5cm}

\usepackage[T1,T2A]{fontenc}
\usepackage[utf8]{inputenc}
\usepackage[english,russian]{babel}

\usepackage{tempora}

\begin{document}

\begin{center}
\MakeUppercase{\textbf{Применение автоматического дифференцирования при
параметрической идентификации стохастических линейных непрерывно-дискретных
систем}} \\
\end{center}

\begin{center} \textbf{
К. К. Горбунов \\
к.т.н., доцент О. С. ЧЕРНИКОВА \\
Новосибирский государственный технический университет \\
г. Новосибирск,} gorbunov.2011@stud.nstu.ru
\end{center}

{\small \noindent Проведен сравнительный анализ методов вычисления градиентов
	целевых функций при численном решении задач нелинейного математического
		программирования и обзор существующих программных средств, реализующих
		автоматическое дифференцирование. Реализована программная процедура
		вычисления критерия максимального правдоподобия и его градиента методом
		автоматического дифференцирования для модели линейной
		непрерывно-дискретной стохастической системы, проведена процедура
		оценивания параметров. \\ The paper provides a survey, including
		bibliography survey, starting from the origins and basics of automatic
		differentiation, continuing by a comparative analysis of overall
		differentiation methods applicable to nonlinear mathematical
		programming problems solving and finishing by an overview of existing
		software tools, frameworks implementing automatic differentiation. A
		software procedure computing the likelihood and its’ gradient using
		automatic differentiation for linear continuous-time dynamic stochastic
		model was implemented as well as parameter estimation based on the
		computed gradient. } \\

Вычисление производных является неотъемлемой частью численного решения
оптимизационных задач градиентными методамию К задачам оптимизации сводятся
многие практические задачи математического моделированияб обработки и анализа
данныхб задачи математической физики \cite{}. Объектами текущих исследований
являются методы и алгоритмы вычисления производныхю Наиболее распространенными
методами вычисления градиентов являются:


\end{document}
